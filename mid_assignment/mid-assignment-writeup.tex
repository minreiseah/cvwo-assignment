% Options for packages loaded elsewhere
\PassOptionsToPackage{unicode}{hyperref}
\PassOptionsToPackage{hyphens}{url}
%
\documentclass[10pt]{exam}

\usepackage{amsmath,amssymb}
\usepackage{lmodern}
\usepackage{iftex}
\ifPDFTeX
  \usepackage[T1]{fontenc}
  \usepackage[utf8]{inputenc}
  \usepackage{textcomp} % provide euro and other symbols
\else % if luatex or xetex
  \usepackage{unicode-math}
  \defaultfontfeatures{Scale=MatchLowercase}
  \defaultfontfeatures[\rmfamily]{Ligatures=TeX,Scale=1}
\fi
% Use upquote if available, for straight quotes in verbatim environments
\IfFileExists{upquote.sty}{\usepackage{upquote}}{}
\IfFileExists{microtype.sty}{% use microtype if available
  \usepackage[]{microtype}
  \UseMicrotypeSet[protrusion]{basicmath} % disable protrusion for tt fonts
}{}
\makeatletter
\@ifundefined{KOMAClassName}{% if non-KOMA class
  \IfFileExists{parskip.sty}{%
    \usepackage{parskip}
  }{% else
    \setlength{\parindent}{0pt}
    \setlength{\parskip}{6pt plus 2pt minus 1pt}}
}{% if KOMA class
  \KOMAoptions{parskip=half}}
\makeatother
\usepackage{xcolor}
\IfFileExists{xurl.sty}{\usepackage{xurl}}{} % add URL line breaks if available
\IfFileExists{bookmark.sty}{\usepackage{bookmark}}{\usepackage{hyperref}}
\hypersetup{
  hidelinks,
  pdfcreator={LaTeX via pandoc}}
\urlstyle{same} % disable monospaced font for URLs
\usepackage{longtable,booktabs,array}
\usepackage{calc} % for calculating minipage widths
% Correct order of tables after \paragraph or \subparagraph
\usepackage{etoolbox}
\makeatletter
\patchcmd\longtable{\par}{\if@noskipsec\mbox{}\fi\par}{}{}
\makeatother
% Allow footnotes in longtable head/foot
\IfFileExists{footnotehyper.sty}{\usepackage{footnotehyper}}{\usepackage{footnote}}
\makesavenoteenv{longtable}
\usepackage{graphicx}
\makeatletter
\def\maxwidth{\ifdim\Gin@nat@width>\linewidth\linewidth\else\Gin@nat@width\fi}
\def\maxheight{\ifdim\Gin@nat@height>\textheight\textheight\else\Gin@nat@height\fi}
\makeatother
% Scale images if necessary, so that they will not overflow the page
% margins by default, and it is still possible to overwrite the defaults
% using explicit options in \includegraphics[width, height, ...]{}
\setkeys{Gin}{width=\maxwidth,height=\maxheight,keepaspectratio}
% Set default figure placement to htbp
\makeatletter
\def\fps@figure{htbp}
\makeatother
\setlength{\emergencystretch}{3em} % prevent overfull lines
\providecommand{\tightlist}{%
  \setlength{\itemsep}{0pt}\setlength{\parskip}{0pt}}
\setcounter{secnumdepth}{-\maxdimen} % remove section numbering
\ifLuaTeX
  \usepackage{selnolig}  % disable illegal ligatures
\fi

\hypersetup{
    colorlinks=true,
    linkcolor=blue,
    filecolor=magenta,      
    urlcolor=cyan,
}


\date{29 Dec 2022}

\begin{document}

\hypertarget{cvwo-assignment}{%
\section{CVWO Assignment}\label{cvwo-assignment}}


\hypertarget{architecture}{%
\subsection{Architecture}\label{architecture}}

\begin{longtable}[]{@{}lll@{}}
\toprule
Name & Description & Technologies \\
\midrule
\endhead
Frontend & UI/UX & Typescript + React \\
Application & RESTful API & Golang (Go-Chi) \\
Database & Persistent Storage Layer & PostgreSQL + Docker \\
Hosting & Deployment, Web Server, Storage & Heroku, Netlify, Render,
AWS(?) \\
\bottomrule
\end{longtable}

\hypertarget{project-requirements}{%
\subsection{Project Requirements}\label{project-requirements}}

\hypertarget{authentication}{%
\subsubsection{Authentication}\label{authentication}}

Authentication allows the forum to control who has access to its
features and content. It provides a personalised and secure experience
for users.

\begin{enumerate}
\def\labelenumi{\arabic{enumi}.}
\tightlist
\item
  \textbf{As a} user, \textbf{I want} to be able to create an account on
  the forum \textbf{so that} I can particiate in discussions(threads)
  and engage with other users.

  \begin{itemize}
  \tightlist
  \item
    The user can create an account by providing a \emph{unique} username
    or email, and password.
  \item
    The user can authenticate through Google or Facebook.
  \end{itemize}
\item
  \textbf{As a} user, \textbf{I want} to be able to log in and log out
  of my account \textbf{so that} I can access my personal settings and
  preferences.

  \begin{itemize}
  \tightlist
  \item
    The user can log in to their account by entering their username and
    password, or authenticate via Google or Facebook.
  \item
    The user can log out of their account by clicking a ``sign out''
    button.
  \end{itemize}
\end{enumerate}

\hypertarget{user-navigation-and-thread-organisation}{%
\subsubsection{User Navigation and Thread
Organisation}\label{user-navigation-and-thread-organisation}}

Good navigation and organisation make the forum easy to use and
accessible to users.

\begin{enumerate}
\def\labelenumi{\arabic{enumi}.}
\tightlist
\item
  \textbf{As a} user, \textbf{I want} to be able to view a list of all
  \emph{recent} threads \textbf{so that} I can have a general overview
  of the forum's content.

  \begin{itemize}
  \tightlist
  \item
    The user can view a list of recent threads organised by time (most
    recent update).
  \end{itemize}
\item
  \textbf{As a} user, \textbf{I want} to be able to view a list of all
  the available threads \textbf{so that} I can \emph{easily} find
  discussions that interest me.

  \begin{itemize}
  \tightlist
  \item
    The user can view a list of all threads topics organised by
    \emph{category}.
  \end{itemize}
\item
  \textbf{As a} user, \textbf{I want} to be able to search for specific
  discussions or keywords \textbf{so that} I can easily find relevant
  information on the forum.

  \begin{itemize}
  \tightlist
  \item
    The user can search for specific discussions or keywords by typing
    their search query into a search bar and clicking a ``search''
    button.
  \end{itemize}
\end{enumerate}

\hypertarget{crud-for-threads-and-posts}{%
\subsubsection{CRUD for Threads and
Posts}\label{crud-for-threads-and-posts}}

The bread and butter of the forum. Only by allowing users to create,
view, update, and delete threads and posts(replies) will the forum allow
for users to particpate in discussion.

\begin{enumerate}
\def\labelenumi{\arabic{enumi}.}
\item
  \textbf{As a} user, \textbf{I want} to be able to create a new thread
  \textbf{so that} I can share my thoughts and ideas with other members.

  \begin{itemize}
  \tightlist
  \item
    The user can create a new thread by providing a title, category, and
    description.
  \end{itemize}
\item
  \textbf{As a} user, \textbf{I want} to be able to view the details of
  a specific thread \textbf{so that} I can see comments and replies.

  \begin{itemize}
  \tightlist
  \item
    The user can view a thread by clicking on a thread.
  \end{itemize}
\item
  \textbf{As a} user, \textbf{I want} to be able to edit or delete my
  comments \textbf{so that} I can correct any mistakes.

  \begin{itemize}
  \tightlist
  \item
    The user can edit or delete their comments by clicking an ``edit''
    or ``delete'' button.
  \end{itemize}
\item
  \textbf{As a} user, \textbf{I want} to be able to reply to other
  people's posts \textbf{so that} I can engage in discussions and
  contribute to the conversation.

  \begin{itemize}
  \tightlist
  \item
    The user can reply to other people's posts by typing their response
    and clicking a ``submit'' button.
  \end{itemize}
\end{enumerate}

\hypertarget{social-features}{%
\subsubsection{Social Features}\label{social-features}}

Social features allow users to interact with each other on the forum.
These features help to foster a sense of community, and encourage
engagement and participation.

\begin{enumerate}
\def\labelenumi{\arabic{enumi}.}
\tightlist
\item
  \textbf{As a} user, \textbf{I want} to be able to upvote or downvote
  other users' comments \textbf{so that} I can show my approval or
  support for another user's comment.

  \begin{itemize}
  \tightlist
  \item
    The user can upvote or downvote other people's posts by clicking the
    respective button next to the comment.
  \end{itemize}
\item
  \textbf{As a} user, \textbf{I want} to be able to customize my profile
  \textbf{so that} I can personalize my experience on the forum.

  \begin{itemize}
  \tightlist
  \item
    The user can customize their profile by clicking on a ``setting''
    button, which allows them to change their profile picture, bio, and
    other personal information.
  \end{itemize}
\item
  \textbf{As a} user, \textbf{I want} to be able to view other users'
  profiles \textbf{so that} I can learn more about them and see their
  past contributions to the forum.

  \begin{itemize}
  \tightlist
  \item
    The user can view other users' profiles by clicking on their
    username, which displays their profile information, and past posts
    and comments.
  \end{itemize}
\end{enumerate}

\hypertarget{moderation-features}{%
\subsubsection{Moderation Features}\label{moderation-features}}

Monitoring and managing content ensures that the forum is a safe place
for users.

\begin{enumerate}
\def\labelenumi{\arabic{enumi}.}
\tightlist
\item
  \textbf{As a} moderator, \textbf{I want} to be able to delete or edit
  posts that violate the forum's rules or guidelines \textbf{so that} we
  can maintain a positive and respectful community.

  \begin{itemize}
  \tightlist
  \item
    The moderator can delete or edit posts by clicking a ``delete'' or
    ``edit'' button next to the post.
  \end{itemize}
\end{enumerate}

\hypertarget{api-endpoints}{%
\subsection{API Endpoints}\label{api-endpoints}}

Refer to
\href{https://github.com/minreiseah/cvwo-assignment/tree/main/api/API.md}{API Documentation}

\hypertarget{execution-plan}{%
\subsection{Execution Plan}\label{execution-plan}}

\hypertarget{dec-9---11}{%
\paragraph{Dec 9 - 11}\label{dec-9---11}}

\begin{itemize}
\tightlist
\item[$\boxtimes$]
  Learn Typescript.
\item[$\boxtimes$]
  Read up on test driven development in React.
\item[$\boxtimes$]
  Design basic wireframes in Figma.
\item[$\boxtimes$]
  Read up on MVC.
\end{itemize}

\hypertarget{dec-12---18}{%
\paragraph{Dec 12 - 18}\label{dec-12---18}}

\begin{itemize}
\tightlist
\item[$\boxtimes$]
  Implement basic frontend in React.
\item[$\boxtimes$]
  Learn about \href{https://quii.gitbook.io/learn-go-with-tests/}{TDD in
  Go}.
\item[$\boxtimes$]
  Read up on \href{https://go-chi.io/\#/README}{Go-Chi} to build REST
  API.
\item[$\boxtimes$]
  Implement database in PostgreSQL.

  \begin{itemize}
  \tightlist
  \item
    Note: Using a postgres:alpine-15 docker image
  \end{itemize}
\item[$\boxtimes$]
  API endpoints scaffolded in Go.
\end{itemize}

\hypertarget{dec-19---30}{%
\paragraph{Dec 19 - 30}\label{dec-19---30}}

\begin{itemize}
\tightlist
\item
  No Work. On Break.
\end{itemize}

\hypertarget{jan-1---7}{%
\paragraph{Jan 1 - 7}\label{jan-1---7}}

\begin{itemize}
\tightlist
\item[$\square$]
  Complete API Endpoints.
\item[$\square$]
  Connect postgres server to Go.
\item[$\square$]
  Dockerise
\end{itemize}

\hypertarget{jan-8---14}{%
\paragraph{Jan 8 - 14}\label{jan-8---14}}

\begin{itemize}
\tightlist
\item[$\square$]
  Hosting.
\item[$\square$]
  Buffer.
\end{itemize}

\hypertarget{jan-15---25}{%
\paragraph{Jan 15 - 25}\label{jan-15---25}}

\begin{itemize}
\tightlist
\item[$\square$]
  Stocktake, stretch goals.
\end{itemize}

\hypertarget{notes}{%
\subsection{Notes}\label{notes}}

\hypertarget{documentation}{%
\paragraph{Documentation}\label{documentation}}

\begin{itemize}
\tightlist
\item
  Go: \texttt{godoc} for docstring
\item
  API: Github pages (tentative)
\end{itemize}

\hypertarget{react}{%
\paragraph{React}\label{react}}

\begin{itemize}
\tightlist
\item
  This project will use npm instead of yarn as I am more familiar with
  the former.
\item
  Chakra UI is chosen over MUI for its flexibility and ease of
  modification.
\item
  I will attempt to integrate Redux from the start rather than
  refactoring my code to integrate it later on.
\end{itemize}

\hypertarget{go}{%
\paragraph{Go}\label{go}}

\begin{verbatim}
api // API information & documentation
build // build files
cmd // contains main.go
internal 
   database // database information
   domain // each domain represents an API subrouter
      categories
         handler.go // endpoint logic
         model.go // type information
         routes.go // subrouter routes
      posts
      threads
      threadsCategories
      users
   router
   server
      routes.go // all main routes
      server.go // server information
\end{verbatim}


\end{document}
