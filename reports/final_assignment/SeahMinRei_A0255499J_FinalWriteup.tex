% Options for packages loaded elsewhere
\PassOptionsToPackage{unicode}{hyperref}
\PassOptionsToPackage{hyphens}{url}
%
\documentclass[12pt]{exam}

\usepackage{amsmath,amssymb}
\usepackage{lmodern}
\usepackage{iftex}
\ifPDFTeX
  \usepackage[T1]{fontenc}
  \usepackage[utf8]{inputenc}
  \usepackage{textcomp} % provide euro and other symbols
\else % if luatex or xetex
  \usepackage{unicode-math}
  \defaultfontfeatures{Scale=MatchLowercase}
  \defaultfontfeatures[\rmfamily]{Ligatures=TeX,Scale=1}
\fi
% Use upquote if available, for straight quotes in verbatim environments
\IfFileExists{upquote.sty}{\usepackage{upquote}}{}
\IfFileExists{microtype.sty}{% use microtype if available
  \usepackage[]{microtype}
  \UseMicrotypeSet[protrusion]{basicmath} % disable protrusion for tt fonts
}{}
\makeatletter
\@ifundefined{KOMAClassName}{% if non-KOMA class
  \IfFileExists{parskip.sty}{%
    \usepackage{parskip}
  }{% else
    \setlength{\parindent}{0pt}
    \setlength{\parskip}{6pt plus 2pt minus 1pt}}
}{% if KOMA class
  \KOMAoptions{parskip=half}}
\makeatother
\usepackage{xcolor}
\IfFileExists{xurl.sty}{\usepackage{xurl}}{} % add URL line breaks if available
\IfFileExists{bookmark.sty}{\usepackage{bookmark}}{\usepackage{hyperref}}
\hypersetup{
  hidelinks,
  pdfcreator={LaTeX via pandoc}}
\urlstyle{same} % disable monospaced font for URLs
\usepackage{longtable,booktabs,array}
\usepackage{calc} % for calculating minipage widths
% Correct order of tables after \paragraph or \subparagraph
\usepackage{etoolbox}
\makeatletter
\patchcmd\longtable{\par}{\if@noskipsec\mbox{}\fi\par}{}{}
\makeatother
% Allow footnotes in longtable head/foot
\IfFileExists{footnotehyper.sty}{\usepackage{footnotehyper}}{\usepackage{footnote}}
\makesavenoteenv{longtable}
\usepackage{graphicx}
\makeatletter
\def\maxwidth{\ifdim\Gin@nat@width>\linewidth\linewidth\else\Gin@nat@width\fi}
\def\maxheight{\ifdim\Gin@nat@height>\textheight\textheight\else\Gin@nat@height\fi}
\makeatother
% Scale images if necessary, so that they will not overflow the page
% margins by default, and it is still possible to overwrite the defaults
% using explicit options in \includegraphics[width, height, ...]{}
\setkeys{Gin}{width=\maxwidth,height=\maxheight,keepaspectratio}
% Set default figure placement to htbp
\makeatletter
\def\fps@figure{htbp}
\makeatother
\setlength{\emergencystretch}{3em} % prevent overfull lines
\providecommand{\tightlist}{%
  \setlength{\itemsep}{0pt}\setlength{\parskip}{0pt}}
\setcounter{secnumdepth}{-\maxdimen} % remove section numbering
\ifLuaTeX
  \usepackage{selnolig}  % disable illegal ligatures
\fi

\hypersetup{
    colorlinks=true,
    linkcolor=blue,
    filecolor=magenta,      
    urlcolor=cyan,
}


\begin{document}

\hypertarget{cvwo-assignment}{%
\section{CVWO Assignment}\label{cvwo-assignment}}

\subsection{User Manual}

To install set up the forum on your local machine, follow these steps:

Option 1 (without Docker)

\begin{enumerate}
\def\labelenumi{\arabic{enumi}.}
\item
  Clone the repository. 
  
  \texttt{git\ clone\ https://github.com/minreiseah/cvwo-assignment.git}
\end{enumerate}


\begin{enumerate}
\def\labelenumi{\arabic{enumi}.}
\setcounter{enumi}{1}
\item
  Install the dependencies: \texttt{cd\ web\ \&\&\ npm\ install}
  (frontend) and \texttt{go\ get} (backend).
\item
  Set up the PostgreSQL database. Migration files are found under
  \texttt{db/migration}.
\item
  Set up the environment variables:

  \begin{enumerate}
  \def\labelenumii{\arabic{enumii}.}
  \item
    \texttt{web/sample.env} (frontend)
  \item
    \texttt{sample.env} (backend)
  \end{enumerate}
\item
  Start the development servers: \texttt{cd\ web\ \&\&\ npm\ start}
  (frontend) and \texttt{make\ start} (backend).\\
\end{enumerate} 

Option 2 (with Docker)

\begin{enumerate}
\def\labelenumi{\arabic{enumi}.}
\item
  Clone the repository.

\texttt{git\ clone\ https://github.com/minreiseah/cvwo-assignment.git}
\end{enumerate}


\begin{enumerate}
\def\labelenumi{\arabic{enumi}.}
\setcounter{enumi}{1}
\item
  Build the Docker images.

\texttt{docker\ compose\ build}
\end{enumerate}


\begin{enumerate}
\def\labelenumi{\arabic{enumi}.}
\setcounter{enumi}{2}
\item
  Set up the environment variables:

  \begin{enumerate}
  \def\labelenumii{\arabic{enumii}.}
  \item
    \texttt{web/sample.env} (frontend)
  \item
    \texttt{sample.env} (backend)
  \end{enumerate}
\item
  Start the Docker containers.

\texttt{docker\ compose\ up}
\end{enumerate}


\hypertarget{reflections}{%
\subsection{Reflections}\label{reflections}}

\hypertarget{diving-in}{%
\subsubsection{Diving in}\label{diving-in}}

I picked Go as my choice of backend for three reasons. One, Ruby seemed
rather outdated and opinionated. Two, Go allows for more flexbility in
building web servers. Three, I wanted to take on the challenge.

After going through this two month process of web development, I can
gladly say that I thoroughly enjoy Go and tolerate React. Go was
surprisingly beginner-friendly with a strong standard library, but
offers a depth of functionality that I am sure to explore in my future
projects.

\hypertarget{testing}{%
\subsubsection{Testing}\label{testing}}

This project involved tests on both the React and Go sides, of which I
am much happier about the latter.

I began my project by building up the frontend. As there was no API to
interface with, my React tests involved mocking axios and simulating API
calls with Jest and the React Testing Library. It was extremely painful
as the tests were not very useful to actually getting a product out nor
were they easy to implement. I simply decided to write these tests
because I wanted to learn how testing in React worked.

The backend, on the other hand, was a lot more enjoyable because I was
actually testing where my SQL queries were functional. For instance, as
categories and threads were tightly coupled via a association table,
writing tests would ensure that any changes made to the category
handlers would not break the thread handlers.

Furthermore, there was a lot more documentation available to writing Go
tests as compared to the ever-changing world of frontend development.
Even chatgpt was not kept up to date with the newest trends in React
development.

If I were ever to be on a team that had to build a frontend independent
of the backend, testing would definitely help to write more readable and
bugfree code. However, for future \emph{personal} projects, I will just
build a backend first to avoid frontend tests. Well, I am still quite
proud of my first attempt at writing unit tests. My next step would be
to write E2E tests.

\hypertarget{challenges}{%
\subsection{Challenges}\label{challenges}}

\hypertarget{docker-hosting}{%
\subsubsection{Docker \& Hosting}\label{docker-hosting}}

While local development was done with docker compose, it is
\emph{relatively} expensive to use in production. Building and deploying
three separate services (frontend, backend, database) via
\texttt{docker\ compose\ up} is not possible (for free) on services such
as AWS because of a requirement to use 1. ECR to host the containers, 2.
ECS to deploy the containers, and 3. RDS to host and manage a PostgreSQL
database.

Furthermore, the above would just allow for the web application to run
on a VPS. However, networking and reverse-proxying would also have to be
done so as to allow end-users to access the application. This is
something I will look into for future projects that require scalability.

Hence, I have resorted to using Render as a \emph{free} cloud hosting
provider for now. For future projects, AWS is definitely something I
would aim to implement.

\hypertarget{networking}{%
\subsubsection{Networking}\label{networking}}

In deciding between hosting all my services on the same network or on
different networks, I went with the latter option for two reasons:

One, simplicity; it is easier to host my services on three different
networks and access them by their external URL. In this way, the PAAS I
used does all the routing.

Two, hosting all services on the same network will take up a lot of time
due to my lack of proficiency with networking.

\hypertarget{what-i-would-have-liked-to-add}{%
\subsection{What I would have liked to
add}\label{what-i-would-have-liked-to-add}}

On documentation; as a novice to frontend development, using React
docgen would have been helpful if not for the constant refactoring and
movement of components. This would have made the project rather
difficult to maintain, requiring more effort to document than to build
the actual product.

After `finishing' the project, I had many API routes written that went
unused. Some features I would have liked to add to the frontend would
be:

\begin{itemize}
\item
  member profile page (view threads and comments)
\item
  inline updating of threads/comments
\end{itemize}

On security, I want to learn how to project my backend API with
authentication. This allows only authorised users to access protected
resources on the server.

\hypertarget{Concluding Remarks}{%
\subsection{Concluding Remarks}\label{concluding-remarks}}

All in all, this project has been a great learning experience. More than
just web development, I gained a sense of appreciation for building
products from a client's point-of-view. I hope to be given the
opportunity to apply my technical skills on projects that make a real
difference in the lives of others.

\end{document}
